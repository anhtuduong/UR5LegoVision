
\documentclass{report}   

\usepackage{graphicx}
\usepackage[labelformat=empty]{caption}
\usepackage{subcaption}

\usepackage{multirow}
\usepackage{graphics,graphicx} % for pdf, bitmapped graphics files
\usepackage{epsfig} % for postscript graphics files
%\usepackage{mathptmx} % assume new font selection scheme installed
\usepackage{times} % assumes new font selection scheme installed
%\usepackage{amssymb}  % assumes amsmath package installed

\usepackage{amsmath} % assumes amsmath package installed
\usepackage{amsfonts}
%\usepackage{amssymb}  % assumes amsmath package installed
\usepackage{url}

\newcommand{\Rnum}{\mathbb{R}} % Symbol fo the real numbers set
\newcommand{\mat}[1]{\ensuremath{\begin{bmatrix}#1\end{bmatrix}}}	% matrix
\newcommand{\myparagraph}[1]{\paragraph{#1}\mbox{}\\}

\begin{document}


\section*{LAB 4: Contact consistent (fixed) base dynamics}


\quad

\noindent
1) \textit{Simulation of contact consistent dynamics:}
Generate a sinusoidal reference for Shoulder Lift  joint (see \textit{ex\_4\_conf.py}) with amplitude 0.6 rad and frequency 1 $Hz$.

\quad

\noindent
2) \textit{Computation of the Constraint Consistent Dynamics:}
Compute the constraint consistent accelerations $\ddot{q}_c$ where a point  contact (i.e. 3D) is possible only at the end-effector (\textit{ee\_link} frame). 

\begin{align}
\ddot{q}_c = M^{-1}[ N_{\bar{M}}(S^T\tau- h ) - J^T\Lambda \dot{J}\dot{q}  ]
\label{eq:ccdyn}
\end{align}

Where $N_{\bar{M}}$ is the null-space projector for $J^{T \#}$. What is the best way to compute the term $\dot{J}\dot{q}$? 
(hint: since $\ddot{x} = J\ddot{q} + \dot{J}\dot{q}$ compute the acceleration at the end-effector while setting $\ddot{q}=0$).

\quad

\noindent
3) \textit{Update of joint velocities after inelastic impact:}
After an unelastic impact the end-effector velocity is going instantaneously to zero $\dot{x}_e =0$. 
An impulse is applied to achieve this, that creates a
discontinuity in the joint velocities. 
Compute the new value of joint velocities that are consistent with the constraint $\dot{x}_e =0$ 
by using the null-space projector $N$ for $J$.

\begin{align}
\ddot{q}^{+} = N \dot{q}^{-} 
\end{align}

%Verify that the linear part of the twist at the end-effector equals $J \dot{q}$.

\quad

\noindent
4) \textit{Simulate the Constraint Consistent Dynamics:}
Use the pre-implemented PD controller (with gravity compensation) and the logic to deal with the contact. 
At the contact consider the appropriate projection of the dynamics \eqref{eq:ccdyn} before 
integrating the accelerations and the velocities (with forward Euler). 


\quad


\noindent
5) \textit{Contact forces disappear when projecting the dynamics:}
Check contact force disappears when after the projection in the null-space of $J^{T \#}$  (i.e. $ N_{\bar{M}}J^Tf = 0$) 
What happens if you use the Moore-penrose pseudo-inverse to compute the projector?
Plot also the torques in the null-space of $J^{T \#}$ during the contact, check that they are barely zero 
 because there is almost no internal joint motion. 
Compare them with a plot of the joint torques to see they are much bigger.
The motion during the contact depends mainly only on the null-space projector $N_{\bar{M}}$
that "cuts-out" the torques that generate contact forces, leaving only the ones that generate internal motions (in this case very small).

\quad

\noindent
6) \textit{Constraint consistent joint reference:}
Try to double the amplitude of the reference of Shoulder Lift joint to 1.2 rad.
Design a reference trajectory that is consistent with the contact (hint: compute $\dot{q}^{d}$ project with $N$ and integrate to get $q^d$).Rember to comment the previous reference generation.
How the reference of the joint changes? it the robot able to move out from the contact?

\quad

\noindent
7) \textit{Gauss principle of least effort}:
Verify that the Gauss principle of least constraint is satisfied (e.g. solve the QP where you minimize the 2-norm distance of the constraint acceleration vector from the unconstraint acceleration one,  under the contact constraint).

\quad

\noindent
8)  \textit{Change in the contact location}:
Modify the code in order to allow the contact at a different location (e.g. origin of \textit{wrist\_3\_link} frame).  Verify that the end-effector now penetrates the ground. 
By activating the \textbf{TF} function in \textit{rviz} you can check the relative location of the frames. 

\quad

\noindent
9) \textit{Check the shifting law (Optional):}
The twist at\textit{ ee\_link} is $v_{e}$, twist at origin of \textit{wrist\_3\_joint} is: $v_o$. Since they belong to the same rigid body (wrist\_3\_link) they are linked the "shifting law",  by a time-invariant \textit{motion} transform ${}_{e}X_o \in \Rnum^{6 \times 6}$:

\begin{align}
  v_{e} = {}_{ee}X_o v_o\\
  J_{e} \dot{q} = {}_{ee}X_o J_o \dot{q}
	\label{fig:}
\end{align}

therefore, also for the Jacobians holds:

\begin{align}
J_{e}  = {}_{e}X_o J_o
\label{fig:}
\end{align}

where: 
\begin{align}
{}_{o}X_e  = \mat{ {}_{o}R_e   &  -{}_o R_e [{}_{e}t]_{\times} \\
					0             &     {}_oR_{e}} 
\label{fig:}
\end{align}

$ {}_{e}X_o =  {}_{o}X_e^{-1}$ and ${}_{e}t$ is the relative position of the origin of frame ${o}$ w.r.t. frame $e$ expressed in frame ${e}$.


 
\end{document}